\documentclass{article}
\usepackage[utf8]{inputenc}

\title{Notes}
\author{dasubermanmind13 }
\date{October 2022}

\begin{document}

\maketitle

Discrete math covers. Combinatorics, sequences, Symbolic logic, and Graph theory
\section{Notes on Logical Connectives}

Logical Connectives

$(P \wedge Q)$ Conjunction. P and Q. This is true when both are True

$(P \lor Q)$ Dis-junction is true when both are True

$(P \rightarrow Q)$ This is an Implication P then Q. Is true when P is false or Q is True or both

$(P \leftrightarrow Q)$ Bi-conditional. Is true when both are True or both are False

$(\neg P)$ Negation

An implication or conditional  statement has the form of....

P is the hypothosis (antecedent) 

Q is the conclusion (consequent)

Most statements in math are implications
For example,

$(a^2 + b^2 = c^2)$

This is only correct if A and B are the legs of a right triangle with a Hypotenuse of C then 
$(a^2 + b^2 = c^2)$ is correct



\end{document}

\end{document}